\chapter{DBpedia Example Input}
\label{appendix:dbpedia}
\singlespace

\section{Long Abstracts}
\begin{lstlisting}[breaklines, basicstyle=\small]
<http://dbpedia.org/resource/Anarchism> <http://dbpedia.org/ontology/abstract> "Anarchism is generally defined as the political philosophy which holds the state to be undesirable, unnecessary, and harmful, or alternatively as opposing authority and hierarchical organization in the conduct of human relations. Proponents of anarchism, known as \"anarchists\", advocate stateless societies based on non-hierarchical voluntary associations. There are many types and traditions of anarchism, not all of which are mutually exclusive. Anarchist schools of thought often differ in fundamental ways, including supporting individualism or collectivism. Schools of anarchist thought are generally divided into the categories of social and individualist anarchism, or similar dual classifications. Anarchism is often considered to be a radical left-wing ideology, and much of anarchist economics and anarchist legal philosophy reflect anti-statist interpretations of communism, collectivism, syndicalism, or participatory economics. Anarchism includes an individualist strain, usually supporting a market economy, private property, and egoism. Some individualist anarchists are also socialists or communists while some anarcho-communists are also individualists. Anarchism as a social movement has regularly endured fluctuations in popularity. The general tendency of anarchism as a social movement has historically been represented by the social anarchist schools anarcho-communism and anarcho-syndicalism, particularly during its early development. Individualist anarchism has historically been primarily a literary phenomenon, although it had a major impact on the social anarchist thought and individualist anarchists have also participated in large anarchist organizations. Most anarchists oppose all forms of aggression, supporting self-defense or non-violence, while others have supported the use of some coercive measures, including violent revolution and propaganda of the deed, on the path to an anarchist society."@en .
<http://dbpedia.org/resource/Achilles> <http://dbpedia.org/ontology/abstract> "In Greek mythology, Achilles was a Greek hero of the Trojan War, the central character and the greatest warrior of Homer's Iliad. Plato named Achilles the most handsome of the heroes assembled against Troy. Later legends (beginning with a poem by Statius in the 1st century AD) state that Achilles was invulnerable in all of his body except for his heel. As he died because of a small wound on his heel, the term Achilles' heel has come to mean a person's principal weakness."@en .
<http://dbpedia.org/resource/A> <http://dbpedia.org/ontology/abstract> "A is the first letter and a vowel in the ISO basic Latin alphabet. It is similar to the Ancient Greek letter Alpha, from which it derives."@en .
<http://dbpedia.org/resource/Albedo> <http://dbpedia.org/ontology/abstract> "Albedo, or reflection coefficient, derived from Latin albedo \"whiteness\" (or reflected sunlight), in turn from albus \"white\", is the diffuse reflectivity or reflecting power of a surface. It is defined as the ratio of reflected radiation from the surface to incident radiation upon it. Being a dimensionless fraction, it may also be expressed as a percentage, and is measured on a scale from zero for no reflecting power of a perfectly black surface, to 1 for perfect reflection of a white surface. Albedo depends on the frequency of the radiation. When quoted unqualified, it usually refers to some appropriate average across the spectrum of visible light. In general, the albedo depends on the directional distribution of incoming radiation. Exceptions are Lambertian surfaces, which scatter radiation in all directions according to a cosine function, so their albedo does not depend on the incident distribution. In practice, a bidirectional reflectance distribution function (BRDF) may be required to characterize the scattering properties of a surface accurately, although the albedo is a very useful first approximation. The albedo is an important concept in climatology and astronomy, as well as in calculating reflectivity of surfaces in LEED sustainable rating systems for buildings, computer graphics and computer vision. The average overall albedo of Earth, its planetary albedo, is 30 to 35%, because of the covering by clouds, but varies widely locally across the surface, depending on the geological and environmental features. The term was introduced into optics by Johann Heinrich Lambert in his 1760 work Photometria."@en .
\end{lstlisting}

\section{Article Titles}
\begin{lstlisting}[breaklines, basicstyle=\small]
<http://dbpedia.org/resource/Alain_Connes> <http://www.w3.org/2000/01/rdf-schema#label> "Alain Connes"@en .
<http://dbpedia.org/resource/Allan_Dwan> <http://www.w3.org/2000/01/rdf-schema#label> "Allan Dwan"@en .
<http://dbpedia.org/resource/Aircraft_Carrier> <http://www.w3.org/2000/01/rdf-schema#label> "Aircraft Carrier"@en .
<http://dbpedia.org/resource/Actress> <http://www.w3.org/2000/01/rdf-schema#label> "Actress"@en .
<http://dbpedia.org/resource/List_of_Atlas_Shrugged_characters> <http://www.w3.org/2000/01/rdf-schema#label> "List of Atlas Shrugged characters"@en .
<http://dbpedia.org/resource/America_the_Beautiful> <http://www.w3.org/2000/01/rdf-schema#label> "America the Beautiful"@en .
<http://dbpedia.org/resource/Ayn_Rand> <http://www.w3.org/2000/01/rdf-schema#label> "Ayn Rand"@en .
<http://dbpedia.org/resource/Alchemy> <http://www.w3.org/2000/01/rdf-schema#label> "Alchemy"@en .
<http://dbpedia.org/resource/Anthropology> <http://www.w3.org/2000/01/rdf-schema#label> "Anthropology"@en .
<http://dbpedia.org/resource/Abacus> <http://www.w3.org/2000/01/rdf-schema#label> "Abacus"@en .
<http://dbpedia.org/resource/Air_Transport> <http://www.w3.org/2000/01/rdf-schema#label> "Air Transport"@en .
\end{lstlisting}

\section{Article Redirects}
\begin{lstlisting}[breaklines, basicstyle=\small]
<http://dbpedia.org/resource/Closest_pair> <http://dbpedia.org/ontology/wikiPageRedirects> <http://dbpedia.org/resource/Closest_pair_of_points_problem> .
<http://dbpedia.org/resource/Silver_Stars_(South_Africa)> <http://dbpedia.org/ontology/wikiPageRedirects> <http://dbpedia.org/resource/Platinum_Stars_F.C.> .
<http://dbpedia.org/resource/Kyeong-yeong_Lee> <http://dbpedia.org/ontology/wikiPageRedirects> <http://dbpedia.org/resource/Lee_Geung-young> .
<http://dbpedia.org/resource/Vichy_Springs> <http://dbpedia.org/ontology/wikiPageRedirects> <http://dbpedia.org/resource/Vichy_Springs,_California> .
<http://dbpedia.org/resource/Culoptila_tarascanica> <http://dbpedia.org/ontology/wikiPageRedirects> <http://dbpedia.org/resource/Culoptila> .
<http://dbpedia.org/resource/Template:Cite_pmid/21359919> <http://dbpedia.org/ontology/wikiPageRedirects> <http://dbpedia.org/resource/Template:Cite_doi/10.1007.2Fs11655-011-0665-7> .
<http://dbpedia.org/resource/Carmen_Bristoliense> <http://dbpedia.org/ontology/wikiPageRedirects> <http://dbpedia.org/resource/Bristol_Grammar_School> .
<http://dbpedia.org/resource/Buenos_Aires_Underground> <http://dbpedia.org/ontology/wikiPageRedirects> <http://dbpedia.org/resource/Buenos_Aires_Metro> .
<http://dbpedia.org/resource/J._w._s._cassels> <http://dbpedia.org/ontology/wikiPageRedirects> <http://dbpedia.org/resource/J._W._S._Cassels> .
\end{lstlisting}

\section{Category Titles}
\begin{lstlisting}[breaklines, basicstyle=\small]
<http://dbpedia.org/resource/Category:Futurama> <http://www.w3.org/2000/01/rdf-schema#label> "Futurama"@en .
<http://dbpedia.org/resource/Category:World_War_II> <http://www.w3.org/2000/01/rdf-schema#label> "World War II"@en .
<http://dbpedia.org/resource/Category:Professional_wrestling> <http://www.w3.org/2000/01/rdf-schema#label> "Professional wrestling"@en .
<http://dbpedia.org/resource/Category:Programming_languages> <http://www.w3.org/2000/01/rdf-schema#label> "Programming languages"@en .
<http://dbpedia.org/resource/Category:Algebra> <http://www.w3.org/2000/01/rdf-schema#label> "Algebra"@en .
<http://dbpedia.org/resource/Category:Anime> <http://www.w3.org/2000/01/rdf-schema#label> "Anime"@en .
<http://dbpedia.org/resource/Category:Abstract_algebra> <http://www.w3.org/2000/01/rdf-schema#label> "Abstract algebra"@en .
<http://dbpedia.org/resource/Category:Linear_algebra> <http://www.w3.org/2000/01/rdf-schema#label> "Linear algebra"@en .
<http://dbpedia.org/resource/Category:Mathematics> <http://www.w3.org/2000/01/rdf-schema#label> "Mathematics"@en .
<http://dbpedia.org/resource/Category:Monarchs> <http://www.w3.org/2000/01/rdf-schema#label> "Monarchs"@en .
\end{lstlisting}

\section{Article Categories}
\begin{lstlisting}[breaklines, basicstyle=\small]
<http://dbpedia.org/resource/Autism> <http://purl.org/dc/terms/subject> <http://dbpedia.org/resource/Category:Autism> .
<http://dbpedia.org/resource/Autism> <http://purl.org/dc/terms/subject> <http://dbpedia.org/resource/Category:Communication_disorders> .
<http://dbpedia.org/resource/Autism> <http://purl.org/dc/terms/subject> <http://dbpedia.org/resource/Category:Mental_and_behavioural_disorders> .
<http://dbpedia.org/resource/Autism> <http://purl.org/dc/terms/subject> <http://dbpedia.org/resource/Category:Neurological_disorders> .
<http://dbpedia.org/resource/Autism> <http://purl.org/dc/terms/subject> <http://dbpedia.org/resource/Category:Neurological_disorders_in_children> .
<http://dbpedia.org/resource/Autism> <http://purl.org/dc/terms/subject> <http://dbpedia.org/resource/Category:Pervasive_developmental_disorders> .
<http://dbpedia.org/resource/Autism> <http://purl.org/dc/terms/subject> <http://dbpedia.org/resource/Category:Psychiatric_diagnosis> .
<http://dbpedia.org/resource/Autism> <http://purl.org/dc/terms/subject> <http://dbpedia.org/resource/Category:Learning_disabilities> .
<http://dbpedia.org/resource/Anarchism> <http://purl.org/dc/terms/subject> <http://dbpedia.org/resource/Category:Anarchism> .
<http://dbpedia.org/resource/Anarchism> <http://purl.org/dc/terms/subject> <http://dbpedia.org/resource/Category:Political_culture> .
<http://dbpedia.org/resource/Anarchism> <http://purl.org/dc/terms/subject> <http://dbpedia.org/resource/Category:Political_ideologies> .
<http://dbpedia.org/resource/Anarchism> <http://purl.org/dc/terms/subject> <http://dbpedia.org/resource/Category:Social_theories> .
<http://dbpedia.org/resource/Anarchism> <http://purl.org/dc/terms/subject> <http://dbpedia.org/resource/Category:Anti-fascism> .
<http://dbpedia.org/resource/Anarchism> <http://purl.org/dc/terms/subject> <http://dbpedia.org/resource/Category:Greek_loanwords> .
<http://dbpedia.org/resource/Agricultural_science> <http://purl.org/dc/terms/subject> <http://dbpedia.org/resource/Category:Agronomy> .
<http://dbpedia.org/resource/Albedo> <http://purl.org/dc/terms/subject> <http://dbpedia.org/resource/Category:Climate_forcing> .
\end{lstlisting}

\section{Category Relationships}
\begin{lstlisting}[breaklines, basicstyle=\small]
<http://dbpedia.org/resource/Category:Middle-earth_languages> <http://www.w3.org/2004/02/skos/core#broader> <http://dbpedia.org/resource/Category:Artistic_languages> .
<http://dbpedia.org/resource/Category:Chess> <http://www.w3.org/2004/02/skos/core#broader> <http://dbpedia.org/resource/Category:Mind_sports> .
<http://dbpedia.org/resource/Category:Philosophers> <http://www.w3.org/2004/02/skos/core#broader> <http://dbpedia.org/resource/Category:Philosophy> .
<http://dbpedia.org/resource/Category:Cereals> <http://www.w3.org/1999/02/22-rdf-syntax-ns#type> <http://www.w3.org/2004/02/skos/core#Concept> .
<http://dbpedia.org/resource/Category:Ordinary_differential_equations> <http://www.w3.org/2004/02/skos/core#prefLabel> "Ordinary differential equations"@en .
<http://dbpedia.org/resource/Category:Snooker> <http://www.w3.org/2004/02/skos/core#broader> <http://dbpedia.org/resource/Category:Precision_sports> .
<http://dbpedia.org/resource/Category:Oceania> <http://www.w3.org/2004/02/skos/core#prefLabel> "Oceania"@en .
<http://dbpedia.org/resource/Category:South_African_people> <http://www.w3.org/2004/02/skos/core#prefLabel> "South African people"@en .
\end{lstlisting}